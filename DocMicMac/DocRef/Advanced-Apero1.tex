\chapter{Advanced orientation}



%=======================================================================================================
\section{Creating a calibration unknown by image}

  % - - - - - - - - - - - - - - - - - - - - -
\subsection{When is it necessary?}

It sometimes happens that each image, or each group of images,
has been acquired with a different set of internal parameters. Here
are possible cases:

\begin{itemize}
    \item when images were acquired with autofocus, which creates variation
          of the focal length (in macro photo, the focal length when focus is
          at $\infty$ is half the focal length when the focus is at image ratio of $1:1$),


    \item when  the image stabilizer is free, this creates (at least)
          variation of principal point;

    \item when  images were acquired with variable zoom.
\end{itemize}


From a photogrammetric point of view, these cases must be avoided
as much as possible; however there are times when the user has no
choice.

  % - - - - - - - - - - - - - - - - - - - - -
\subsection{Examples}

The directory {\tt applis/XML-Pattron/Oiseau-Margot/} contains XML
files that were used to process images acquired with macro lenses.
The files {\tt New-Apero1.xml} to {\tt New-Apero6-ExportDirPlanMM.xml}
contain examples on real cases of the different mechanisms shown
here. By the way, these examples may be a bit complex because they
were made before key standardization.

See especially the examples {\tt Apero-ExCalibPerIm-1.xml} and
{\tt Apero-ExCalibPerIm-2.xml} on {\tt MurSaintMartin}, that have
been added after writing this section; they are more realistic
and contain some comments at the beginning that should be
sufficient.

  % - - - - - - - - - - - - - - - - - - - - -
\subsection{How to create unknowns}

The optional section {\tt <CalibPerPose>}, under {\tt <CalibrationCameraInc>},
allows to handle these cases. It contains a mandatory args {\tt   <KeyPose2Cal>}:


\begin{itemize}
    \item it must contain a string $K$ which describes  a key association,
          (see~\ref{Ref:Key:Map}  for {\tt <KeyedNamesAssociations>});
    \item  two images  $I_1$ and $I_2$ will share the same internal parameters,
           if and only if $K(I_1) = K(I_2)$;
     \item for example,  if $K$ is the identity key, a new calibration will
           be created for each new image;

     \item  for each of these calibrations, the identifier will be $K(I)$ and not,
            as usual, the tag {\tt <Name>}; this is necessary because elsewhere
             different internal calibration would have the same identifier;

      \item this identifier is used when it is necessary to refer
            to a set of internal calibration (for example when applying
            constraint only to a subset of the existing calibration);

\end{itemize}

  % - - - - - - - - - - - - - - - - - - - - -
\subsection{Saving  results with  variable calibrations}

The most current case for using these mechanisms is when there is one calibration
per image. In this case, the easiest way for handling the results is  to
simply  save the internal calibration  with the external calibration;
this is, by default, what Apero does  in {\tt <ExportPose>}, see
{\tt Apero-0.xml} in~\ref{Result:Apero0}.


For more complicated cases,  an {\tt <ExportCalib>} section  will have to be
used with the following tags:

\begin{itemize}
    \item {\tt <PatternSel>}  to specify to which calib it applies; the selection
         is made on the identifier (here $K(I)$);

    \item {\tt <KeyAssoc>}  to specify how to compute a file name from the identifier;

    \item {\tt <KeyIsName>}  at {\tt false} \footnote{which is, however, the default value} meaning
          that {\tt <KeyAssoc>}  is a key and not an absolute name;
\end{itemize}


  % - - - - - - - - - - - - - - - - - - - - -
\subsection{Loading initial values  with  variable calibrations}

When  variable internal calibration is used a first time,
the different calibration can be initialized with the same
value.  This value can be read as usual in the {\tt <NameFile> } of {\tt <CalValueInit>}.
See an example in {\tt  applis/XML-Pattron/Oiseau-Margot/New-Apero1.xml},

There is other cases where it will be needed to initialize these  calibrations
from variable values; for example, values  that have been computed and saved by a
previous running of Apero. In this case, a {\tt KeyedNamesAssociations} will be used
to specify the association between \emph{the  name of the pose} and the file where
the initial value is to be read; this key is specified in the optional
{\tt <KeyInitFromPose>}; see example in {\tt  applis/XML-Pattron/Oiseau-Margot/New-Apero3.xml}.


  % - - - - - - - - - - - - - - - - - - - - -

\subsection{Examples with group of poses}

Sometimes we do not want to create a calibration for each image,
but a calibration for each group of images. This can occur
when we know that the parameters affecting the calibration
(zoom, focus) have changed, but only a few times, and we are able
to specify which groups of images share the same parameters.
This can also occur when we have used several versions of the
same camera with the same focal length (so not distinguishable
with the procedure used in {\tt Tapas}).

See {\tt Apero-ExCalibPerGROUPIM-1.xml} and
{\tt Apero-ExCalibPerGROUPIM-2.xml}  on  {\tt MurSaintMartin} data
set, it illustrates how this can be done. The file contains
some comments.


The files {\tt Apero-ExCalibPerIm-1.xml} and {\tt Apero-ExCalibPerIm-2.xml}
on  {\tt MurSaintMartin} have been added and contain also
examples for calibration per images which are probably easier to understand
than {\tt  applis/XML-Pattron/Oiseau-Margot/}.


  % - - - - - - - - - - - - - - - - - - - - -


\subsection{Enforcing a smooth evolution}

Someting, it may be usefull to have a calibration per image but to also
enforce that this calibration evolve slowly. This may be the case,
for example, if the  variation of focal lenght is due to thermal evolution.

This is possible with option {\tt  ContrCalCamCons} of {\tt Campari}, for now
the calibration must be of type {\tt ModelePolyDeg0} or {\tt ModelePolyDeg1}
\footnote{This can be easily generalized, but not sure it is usefull}.

The file {\tt Cmd2.txt} in {\tt Documentation/Data/Rigid-Block} contains an example. The two last command are  :

\begin{verbatim}

Tapas  AddPolyDeg1 DSCF.*jpg InOri=Ori-AllRel/

Campari DSCF.*jpg Ori-AddPolyDeg1/ TestSmoothEvolv CPI1=true ContrCalCamCons=[Loc-Assoc-Im2Block,2] FocFree=1 PPFree=1  NbIterEnd=8
\end{verbatim}

Some comments :

\begin{itemize}
   \item   we first use {\tt Tapas AddPolyDeg1} to make a $1$  degree polynomial model;
   \item   in {\tt Campari} we use the {\tt CPI1=true} to  make one calibration per image (else is would be useless);
   \item    we use also {\tt FocFree=1 PPFree=1} to free focal and principal point (else again the option would be useless)
   \item     the option {\tt ContrCalCamCons} contains two value, first is a key (here {\tt Loc-Assoc-Im2Block})
             , second is sigma (here $\sigma=2$ meaning we expect the focal to evolve $\pm$ 2pix from calibration to calibration);
\end{itemize}

The meaning of {\tt Loc-Assoc-Im2Block} is :

\begin{itemize}
    \item it must return for each name of image, two value {\tt Time} and {\tt Grp};
    \item the value {\tt Grp} indicate the group of image, the constraint will be applied inside
          image of the same {\tt Grp};
    \item the value {\tt Time} indicate an order, the constraint will be applied between pair of
          consecutive camera regarding this order;
    \item  here the key that describe the rigid block works perfectly for what we want to do.
\end{itemize}

The meaning of $\sigma$ is :

\begin{itemize}
   \item let $C_1$ and $C_2$ be  two succesive calibration
   \item for a pixel $P$ let $N_k(P)$ be the direction of emerging ray for camera $C_k$
   \item we add to the bundle adjustment minimisation function a regularisation equation $R{1,2} $ as writes  \ref{Eq:Cons:Cam}
   \item  $\iint \limits_{C}$ mean integral on the whole sensor;
   \item  $ \sigma$ is expressed in pixel.
\end{itemize}

\begin{equation}
   R_{1,2} = \frac{\iint \limits_{C} |N_1(P)-N_2(P)|^2} {\iint  \limits_{C} \sigma ^2} 
  \label{Eq:Cons:Cam}
\end{equation}

{\tt Campari} has printed an additionnal message :


\begin{verbatim}
ContCamConseq= 9.69226e-06 for DSCF3297_L.jpg
ContCamConseq= 2.19443e-05 for DSCF3297_R.jpg
ContCamConseq= 2.0443e-05 for DSCF3298_L.jpg
ContCamConseq= 2.14662e-05 for DSCF3298_R.jpg
\end{verbatim}

These value are the residual of formula $ R_{1,2}$. Here they are very low,
in fact we can check the computed value give almoste the same focals :

\begin{verbatim}
grep "<F>"  Ori-TestSmoothEvolv/Orientation-DSCF329*
Ori-TestSmoothEvolv/Orientation-DSCF3297_L.jpg.xml:               <F>4332.52026468588519</F>
Ori-TestSmoothEvolv/Orientation-DSCF3297_R.jpg.xml:               <F>4374.50135394254175</F>
Ori-TestSmoothEvolv/Orientation-DSCF3298_L.jpg.xml:               <F>4332.52029293327632</F>
Ori-TestSmoothEvolv/Orientation-DSCF3298_R.jpg.xml:               <F>4374.50128551130183</F>
Ori-TestSmoothEvolv/Orientation-DSCF3299_L.jpg.xml:               <F>4332.52035901077033</F>
Ori-TestSmoothEvolv/Orientation-DSCF3299_R.jpg.xml:               <F>4374.50121606732682</F>
\end{verbatim}





%=======================================================================================================

\section{Database of existing calibration}

\label{DB:Calib}
\subsection{General points}


%=======================================================================================================
\section{Auxiliary exports}

  % - - - - - - - - - - - - - - - - - - - - -

\subsection{Generating point clouds with {\tt <ExportNuage>}}

\label{Ap:Exp:Nuage}


%=======================================================================================================

\section{Using scanned analog images}

\subsection{Dealing with internal orientation}

\label{Analog:Image}
Scanned analog images are important in many applications, as they represent
a valuable source of information for studying phenomena on long periods of time.
From the photogrammetric point of view, the main difference between
scanned images and digital camera is that for each images there
is a specific  transformation between the photo and the scanner.
This transformation can be computed when there exist fiducial marks
on the camera. This section presents how this can be done with {\tt Apero/MicMac}.


The following link \url{http://micmac.ensg.eu/data/DemoScanned_Dataset.zip} contains a data set that illustrates these
features. It contains $5$ images that are a simulation of scanned
images: the images have been randomly rotated and scaled, simulating the
interior orientation of the scanner; before the rotation, $8$  fiducial marks have
been added. Figure~\ref{ImFid} illustrates this data set.


\begin{figure}
\begin{tabular}{||c|c||}
   \hline \hline
   \multicolumn{2}{c|}{\includegraphics[width=120mm]{FIGS/Niche/ImGlob.jpg}} \\
   \includegraphics[width=40mm]{FIGS/Niche/Cible1.jpg} &
   \includegraphics[width=40mm]{FIGS/Niche/Cible2.jpg}
\end{tabular}
\caption{Simulation of fiducial marks: an image and two marks }
\label{ImFid}
\end{figure}


There are two slight differences in data processing between such data sets and "classical" digital images:


\begin{itemize}
   \item the position of fiducial mark on images and on the camera has to be indicated;
   \item the calibration will not be expressed in pixel, but in the same unit
         as the position of the reference fiducial marks (generally $mm$).
\end{itemize}

On {\tt DemoScanned/}, the directory {\tt Ori-InterneScan/} contains all the information
about the fiducial marks. It works like this:

\begin{itemize}
    \item each file contains a structure, a type {\tt <MesureAppuiFlottant1Im>} which describes a list
          of named points (here the names are {\tt P0, P1, \dots}; this is the same structure as the one used
          for image measurement of GCP, as seen in~\ref{GCP:Org};

     \item there is a file {\tt MeasuresCamera.xml} that contains the position of the fiducial marks
          on the camera;

     \item for each image {\tt XXX}, there is a file {\tt MeasuresIm-XXX.xml} that contains the position
           of the marks on the image; when this file does not exist, the image is considered to be a "classical"
           digital image that will be processed as usual;

     \item if required\footnote{for example if several analog camera are used in the same bundle},
           it is possible to change the association between an image and the two files: position in camera
           and image; for this, you must change the  value of {\tt Key-Assoc-STD-Orientation-Interne}
           \footnote{see in include/XML\_GEN/DefautChantierDescripteur.xml the default value}
           in your {\tt MicMac-LocalChantierDescripteur.xml}
\end{itemize}

Here, the file {\tt MeasuresCamera.xml} contains the positions of fiducial marks in $mm$,
all the calibration  must then be have the same unit and be in the same frame than these
marks. For technical reasons, this point of reference must be the upper left corner and not the center.
As {\tt Apero/MicMac} cannot deal correctly with default calibration in $mm$, we
have to give an initial value in file {\tt Ori-CalibInit}. Some comments on this file:

\begin{itemize}
    \item  the size of image {\tt <SzIm>} is also in $mm$ (as the normalization focal
            {\tt <Etats>}  and all parameters).
     \item  the optional {\tt  <ScannedAnalogik> } is set to true, required because
            it will indicate to {\tt Apero} to be "tolerant" if tie points are detected out
            of the bounding box $[0,0]x[24,36]$;
\end{itemize}

The file {\tt ExCmd.txt} contains a possible processing of the data:


\begin{itemize}
    \item {\tt Tapioca  All ".*jpg"  1200}, as usual \dots
    \item {\tt Tapas  FishEyeBasic  ".*jpg"  Out=Ori1 InCal=CalibInit PropDiag=0.68}

     \begin{itemize}
          \item  it is necessary to indicate the calibration in {\tt CalibInit} in $mm$ because {\tt Apero}
                 would not built it correctly;
          \item no need to indicate situation of fiducial mark, the def value of {\tt Key-Assoc-STD-Orientation-Interne}
                will force {\tt Apero} to look for them at the right place;
           \item {\tt PropDiag=0.68} , because it is a hemispheric fisheye;
      \end{itemize}

    \item {\tt Tapas FishEyeBasic ".*jpg" InOri=Ori1  Out=Ori2 PropDiag=0.68}
    \begin{itemize}
           \item  just to check that {\tt Tapas} can be iterated in  this  configuration
    \end{itemize}
    \item {\tt Malt GeomImage ".*jpg" Ori2 Master=IMG\_5693\_Out.jpg Spherik=true SzW=2}
    \begin{itemize}
             \item the {\tt Spherik=true} is well adapted  to the scene, in this geometry
                   {\tt MicMac} computes the depth  $R=f(i,j)$ where $R$ is the distance
                   to the master image center \footnote{i.e. rectification is made on sphere};
    \end{itemize}

    \item {\tt Nuage2Ply MM-Malt-Img-IMG\_5693\_Out/NuageImProf\_STD-MALT\_Etape\_8.xml Attr=IMG\_5693\_Out.jpg Scale=2}

    \begin{itemize}
             \item usual generation of point cloud in ply format (figure~\ref{ImNiche});
    \end{itemize}

\end{itemize}




\begin{figure}
   \includegraphics[width=120mm]{FIGS/Niche/snapshot00.jpg}
\caption{Point cloud with simulation of scanned images}
\label{ImNiche}
\end{figure}


If you take a look at orientation files, you will see that they are self sufficient for
matching: the {\tt <OrIntImaM2C>}  section contains the affinity between scanner and image computed
from the fiducial marks:

\begin{verbatim}
     <OrientationConique>
...
          <OrIntImaM2C>
               <I00>2753.06179948014324 1771.95961861625119</I00>
               <V10>-72.5948450058228758 4.58183503308403939</V10>
               <V01>-4.5818350330840607 -72.5948450058228332</V01>
          </OrIntImaM2C>
...
     </OrientationConique>
\end{verbatim}


%=======================================================================================================

\subsection{Semi-automatic fiducial mark input with Kugelhupf}

Kugelhupf (Klics Ubuesques Grandement Evites, Lent, Hasardeux mais Utilisable pour Points Fiduciaux) is a tool for scanned images. It is used to automatically find fiducial marks on the images.

If the fiducial marks are almost on the same position on each picture, it is necessary to point them on one image and Kugelhopf will find the fiducial marks on the others.

Syntax of {\tt Kugelhupf}:
\begin{verbatim}
*****************************
*  Help for Elise Arg main  *
*****************************
Mandatory unnamed args : 
  * string :: {Pattern of scanned images}
  * string :: {2d fiducial points of an image}
Named args : 
  * [Name=TargetHalfSize] INT :: {Target half size in pixels (Def=64)}
  * [Name=SearchIncertitude] INT :: {Search incertitude in pixels (Def=5)}
  * [Name=SearchStep] REAL :: {Search step in pixels (Def=0.5)}
\end{verbatim}

Call example:
{\tt mm3d Kugelhupf "1987\_FR4074.*.tif" Ori-InterneScan/MeasuresIm-202.tif.xml SearchIncertitude=10 }

The output is xml files in {\tt Ori-InterneScan} for every picture where the automatic search was successful.
If at least one point was not found for an image, the xml file is not created.
Kugelhupf only works on picture that have no xml file.

This is useful to make successive calls to Kugelhupf with different search incertitudes :

{\tt mm3d Kugelhupf "1987\_.*.tif" Ori-InterneScan/MeasuresIm-202.tif.xml SearchIncertitude=10}

{\tt mm3d Kugelhupf "1987\_.*.tif" Ori-InterneScan/MeasuresIm-202.tif.xml SearchIncertitude=25}


The first call will be fast and will have a result only with pictures with close fiducial points, and the second call will
be slower but only on the worst pictures.


\subsection{FFT variant with  FFTKugelhupf}

The {\tt FFTKugelhupf} tool is a variant of {\tt Kugelhupf}; the "philosophy" and interface is
the same than  {\tt Kugelhupf} : it assumes that the mark can be retrieved from a "master" image 
that gives  the shape and approximate position of each mark. When the interval of research
is important {\tt FFTKugelhupf} can be significativelly faster due its use of fast fourrier transform
for initial guess and multi resolution for more accurate positionning. However the two tool
are of interest as there is no much testing of   {\tt FFTKugelhupf} which may fail more frequently
when the mark are very small.

We can test it with the data set DemoScaned of~\ref{Analog:Image}.

\begin{verbatim}
mm3d FFTKugelhupf "IMG_569[4-7]_Out.jpg" Test-5963.xml  Masq=NONE 
ESIDU = 94.3549 for IMG_5697_Out.jpg
RESIDU = 0.269043 for IMG_5696_Out.jpg
RESIDU = 300.811 for IMG_5694_Out.jpg
RESIDU = 144.772 for IMG_5695_Out.jpg
\end{verbatim}

Note :

\begin{itemize}
    \item  {\tt RESIDU = 0.269043} is the value of the residual computed by adapting an affinity
          between the initial mark and the detected mark; it is a good  indicator of the quality
          of the match;
    \item  here with this unreallistic data set, the resul are rather poor with only one image with all good matches;
    \item with a data set more realistic as illustrated in figure~\ref{ImFidMark} , we obtain currently $100\%$ of images
          with residual better than one pixel;
    \item  the default interval are quite high $500$ pixel for the incertitude and $150$ for the target half size;
\end{itemize}

\begin{figure}
   \includegraphics[width=60mm]{FIGS/Niche/Fid1.jpg}
   \includegraphics[width=60mm]{FIGS/Niche/Fid2.jpg}
\caption{Real Fiducial mark}
\label{ImFidMark}
\end{figure}


\begin{verbatim}
*****************************
*  Help for Elise Arg main  *
*****************************
Mandatory unnamed args : 
  * string :: {Pattern of scanned images}
  * string :: {2d fiducial points of an image}
Named args : 
  * [Name=TargetHalfSize] Pt2di :: {Target half size in pixels (Def=150)}
  * [Name=Masq] string :: {Masq extension for ref image, Def=Fid, NONE if unused}
  * [Name=SearchIncertitude] INT :: {Def=500}
  * [Name=SzFFT] INT :: {Sz of initial fft research, power of recomanded, Def=256 or 128 depending other}
\end{verbatim}

Two paramaters are specific to {\tt FFTKugelhupf} :

\begin{itemize}
   \item {\tt Masq} , if it exist it must be a mask superposable to the master image , the correlation is then
         restricted to this mask
   \item {\tt SzFFT} size of the initial reduced image on which computation is made using fast fourrier transform; 
         with all other parameters set to their default values, it is set to $128$ which makes a resolution decimation of $10$;
\end{itemize}

\subsection{Resampling images with {\tt ReSampFid}}

This section describes an approach to analog image processing different from the one described in~\ref{Analog:Image}.
In~\ref{Analog:Image} the image are unchanged and an intenal orientation is computed. Here conversely the image are
resampled in a geometry where all the mark are superposable, then the resampled images can be used as "standard" 
images acquired by difgital camera.

On one hand, the approach of~\ref{Analog:Image} is theoretically slightly better as it avoid
 one resampling of the image. On the other hand, the approach
described here is simpler for the implementation as once the first resampling is done, 
there no more special case to deal with.
Practically the approach of~\ref{Analog:Image} has led to severally tricky bug and, and it is highly recommanded to use
the approach described here as we probably will not have the man power to correct the next problem that may/will occur
with the internal orientation approach.

\begin{verbatim}
mm3d ReSampFid -help
*****************************
*  Help for Elise Arg main  *
*****************************
Mandatory unnamed args : 
  * string :: {Pattern image}
  * REAL :: {Resolution}
Named args : 
  * [Name=BoxCh] Box2dr :: {Box in Chambre (generally in mm)}
  * [Name=Kern] INT :: {Kernel of interpol,0 Bilin, 1 Bicub, other SinC (fix size of apodisation window), Def=5}
\end{verbatim}

Some comments :

\begin{itemize}
   \item  The value for fiducial mark position, on image and in the chambre, are searched in {\tt Ori-InterneScan/},
          or more exactly on the file described by the key {\tt Key-Assoc-STD-Orientation-Interne};

   \item the first parameter is the pattern of image to resample, if there is more than one, the process is done in parallel;

   \item the second parameter is the resampling resolution; for example if the fiducial mark are expressed in mm , and the
         image where scanned at $20$ micron, a reasonable value could be $0.02$;

   \item if the {\tt BoxCh} is omitted, a default value will be computed using the englobing box of the fiducial marks;

   \item the image are renamed using the rule {\tt toto.tif} $\rightarrow$  {\tt OIS-Reech\_toto.tif} , with
         image name begining by {\tt OIS-Reech},  the  default rules ignorate the fiducial marks, so these images
         can be used as is; if user want to rename the resampled images, then he will need to hide the
         {\tt Ori-InterneScan/} folder;

\end{itemize}



%=======================================================================================================
\section{Adjustment with lines} \label{Adjustment with lines}



\subsection{Introduction}

This section describes the case where user wants to adjust orientation using correspondence between $3d$ points
and image lines. It's rather specific, and was added in the context of georeferencing aerial photographs with a $3d$
data base of roads \footnote{I am not sure to see if there exist other examples of application}. Formally we have:

\begin{itemize}
    \item a set of $2d$ line $L_k$ in images. Let $I_k$ be the corresponding image and $\pi_k$ be the projection
         function that goes from ground coordinates to $I_k$;
    \item for each $3d$ point a set of point $p_{k,i}$ and we "know" that $\forall k,i \, : \pi_k (p_{k,i}) \in L_k $
    \item let $D(p,L)$ be the distance between point $p$ and line $L$.
\end{itemize}

Mathematically we want to add the cost to the global minimization:

\begin{equation}
    \sum_{k,i} D^2 ( \pi_k(p_{k,i}),L_k) \label{EQU:LINE:POINT}
\end{equation}


        % - - - - - - - - - - - - - - - - - - - - - - - - - - - - - - - - - - - - - - -

\subsection{Data set}

The folder {\tt Documentation/Data/CompensOnLine/} contains  data that illustrates how this can be done in
{\tt Apero}. These are completely artificial (simulated) data which were generated for developing and testing this feature: in this example it will be possible to cancel completely the equation~\ref{EQU:LINE:POINT} which will obviously not
be the case with real data.

To run the data set extracted from the mercurial server, one first needs to compute the tie points with:

\begin{itemize}
   \item {\tt mm3d Tapioca All Abbey-IMG\_020.*jpg 1200}
\end{itemize}

To test all features, this data set has been processed as if it was the scans of analog images
(see {\tt Ori-InterneScan/}), but it also works with digital images. As it can be seen in {\tt mm3d-LogFile.txt},
the orientation has been computed with the command:

\begin{itemize}
   \item {\tt mm3d Tapas AutoCal Abbey-IMG\_020.*jpg InCal=Ori-CalibInitAnalogik/ Out=Rel}

\end{itemize}

This toy example is made from $4$ small images on the same strip. Of course the data are completely  artificial:

\begin{itemize}
   \item when using the "real" orientation (i.e Ori-Rel/) the projection of point on lines is perfect;
   \item there exist data for each image;
   \item for each images the data is sufficient to compute its orientation;
\end{itemize}

        % - - - - - - - - - - - - - - - - - - - - - - - - - - - - - - - - - - - - - - -

\subsection{Organization of information}

Basically, the information is structured the same way as for GCP in~\ref{GCP:Org},\ref{Sec:GCPBascule} and
\ref{CAMPARI} :

\begin{itemize}
   \item there is a file for storing the $3d$ point, in this example {\tt MesurePointGround.xml}, it is the same
         structure that for {\tt GCP}, and the adjustment of this point can mix measurement of $3d$ in ground,
         $2d$ points in images and $2d$ lines in images;

   \item there is a file for storing $2d$ images line coordinates and the points that are supposed to project
         on these lines, in this example {\tt MesureLineImage.xml};

   \item in {\tt Apero} these files are loaded, linked and then used for adjustment.
\end{itemize}

The structure of {\tt MesureLineImage.xml} should be quite obvious:

\begin{itemize}
   \item it must contain a global structure {\tt <SetOfMesureSegDr>};
   \item the {\tt <SetOfMesureSegDr>} is a list of {\tt <MesureAppuiSegDr1Im>}, each one
         storing the observation related to one image;
   \item a  {\tt MesureAppuiSegDr1Im} contains the name of the image {\tt <NameIm>}, and list a of {\tt <OneMesureSegDr>}, each one storing the information related to a line;

   \item a   {\tt <OneMesureSegDr>} contains exactly two $2$ points {\tt <Pt1Im>} and {\tt <Pt2Im>} that store
            the geometry of the line and a list of {\tt <NamePt>} that contains the name of the $3d$ points;
            there can be any number $N \geq 1$ of  {\tt <NamePt>} , even if in this example we
             have $N=2$ everywhere
             \footnote{which will probably be the standard case when the $3d$ points come from $3d$ lines}.

\end{itemize}

        % - - - - - - - - - - - - - - - - - - - - - - - - - - - - - - - - - - - - - - -

\subsection{Example {\tt Apero-2-DroiteStatique.xml}}

\label{Apero-2-DroiteStatique}

It illustrates file loading.
In this example, we only load the observation and check that the projection is perfect (because data
are simulated). It's pretty much the same than~\ref{GCP:Org} :

\begin{itemize}
   \item {\tt BDD\_ObsAppuisFlottant} load the line information, in a data base name {\tt Id-Appui};
         the name of the file containing the line is stored in {\tt  <KeySetSegDroite>};
   \item {\tt PointFlottantInc} load the $3d$ points information, in the same data base {\tt Id-Appui};
   \item {\tt ObsAppuisFlottant} add the observation of the data base {\tt Id-Appui} to the adjustment.
\end{itemize}

As in this example the data are all frozen \footnote{see {\tt ePoseFigee} and {\tt eAllParamFiges}}, we just print the
value of the observed distance, which turns to b $0$ up to the rounding error. We obtain the error for each point, and
the maximum of error :


\begin{verbatim}
...
    ErrMax = 1.65784e-11 For I=Abbey-IMG_0205.jpg,  C=P_8_B pixels
  - - - - - - - - - - -
==== ADD Pts P_9_A Has Gr 1 Inc [1,1,1]
--NamePt P_9_A Ec Estim-Ter [0,0,0]           Dist =0 ground units
Inc = [1,1,1]PdsIm = [1,1]
    Ecart Estim-Faisceaux 16.2788
      ErrMoy 2.71776e-13 pixels  SP=2
     ErrMax = 2.71776e-13 For I=Abbey-IMG_0204.jpg,  C=P_9_A pixels
  - - - - - - - - - - -
==== ADD Pts P_9_B Has Gr 1 Inc [1,1,1]
--NamePt P_9_B Ec Estim-Ter [0,0,0]           Dist =0 ground units
Inc = [1,1,1]PdsIm = [1,1]
    Ecart Estim-Faisceaux 16.1819
      ErrMoy 6.65003e-12 pixels  SP=2
     ErrMax = 8.11932e-12 For I=Abbey-IMG_0205.jpg,  C=P_9_B pixels
  - - - - - - - - - - -

   ============================= ERRROR MAX PTS FL ======================
   ||    Value=4.56508e-10 for Cam=Abbey-IMG_0204.jpg and Pt=P_2_A
   ======================================================================
\end{verbatim}

        % - - - - - - - - - - - - - - - - - - - - - - - - - - - - - - - - - - - - - - -

\subsection{Example {\tt Apero-3-DroiteEvolv.xml}}

In this example, we  want to check that with "perfect" data, the adjustment on line
is sufficient to compute the "perfect" orientation as long as we are reasonably initialized.
It's pretty much the same than~\ref{Apero-2-DroiteStatique},
except that we start from start from orientation that have been modified and
do not freeze the orientation.


\begin{verbatim}
...
  ============================= ERRROR MAX PTS FL ======================
   ||    Value=16.9249 for Cam=Abbey-IMG_0207.jpg and Pt=P_13_A
   ======================================================================
--- End Iter 0 ETAPE 0
...
...
   ============================= ERRROR MAX PTS FL ======================
   ||    Value=0.00335622 for Cam=Abbey-IMG_0207.jpg and Pt=P_13_A
   ======================================================================
--- End Iter 1 ETAPE 0
...
...
   ============================= ERRROR MAX PTS FL ======================
   ||    Value=8.87999e-09 for Cam=Abbey-IMG_0204.jpg and Pt=P_14_B
   ======================================================================
--- End Iter 2 ETAPE 0
...
...
   ============================= ERRROR MAX PTS FL ======================
   ||    Value=3.17978e-10 for Cam=Abbey-IMG_0204.jpg and Pt=P_2_A
   ======================================================================
--- End Iter 3 ETAPE 0
...
...
   ============================= ERRROR MAX PTS FL ======================
   ||    Value=3.17775e-10 for Cam=Abbey-IMG_0204.jpg and Pt=P_2_A
   ======================================================================
--- End Iter 4 ETAPE 0

\end{verbatim}

At the end, the orientation are exported in {\tt Ori-Check3/}, and it can be seen that they are almost identical to {\tt Ori-Rel}


        % - - - - - - - - - - - - - - - - - - - - - - - - - - - - - - - - - - - - - - -

\subsection{Example {\tt Apero-4-CompensMixte.xml} and  {\tt  Apero-5-CompensAll.xml}}

In this example, we check that, as we would do in real case, it is possible to adjust simultaneously GCP-line with
other measurement. Also we use a reduced set of line-point (with {\tt MesureLineImageIncompl.xml}), in this set
there is lines only for the two central images.

In {\tt Apero-4-CompensMixte.xml} we check that using simultaneously tie point and "GCP-line", we can recover the orientation
of the extreme images (where there is no GCP-line).

{\tt  Apero-5-CompensAll.xml} had a minor modification, we see that in {\tt <BDD\_ObsAppuisFlottant >} we have

\begin{itemize}
   \item  {\tt <KeySetSegDroite> } as before , to make adjustment on line;
   \item an additional {\tt  <KeySetOrPat>} to make adjustment of $2d$ points like in  ~\ref{GCP:Org};

\end{itemize}

So in this example, the $3d$ point {\tt P\_1\_A} will be adjusted simultaneously on:

\begin{itemize}
    \item $2d$ point measure stored in {\tt MesurePointImagePart.xml};
    \item $2d$ line measure stored in {\tt MesureLineImage.xml};
    \item $3d$ point measure stored in {\tt MesurePointGround.xml}.
\end{itemize}

%=======================================================================================================
\section{Recent evolution in Tapas and other orientation tools}

The evolution described here, are now integrated in the "standard" {\tt Tapas} command.
When necessary \footnote{for example because "new"  {\tt Tapas} has slower convergence}, it is possible
to get back to the previous behavior by using the {\tt OldTapas} command.



        % - - - - - - - - - - - - - - - - - - - - - - - - - - - - - - - - - - - - - - -

\subsection{Viscosity \& Levenberg Marquardt stuff}

Adding viscosity, also named more pedantically using Levenberg Marquardt algorithm,
is a classical way to avoid problem due to ill conditioning system in energy minimization algorithm.
In theory the only drawback is that it slow down the speed of convergence.

However, in Tapas, the default value of viscosity was badly dosed, and it  some configuration ,
particularly UAV acquisition \footnote{because it's big data set and the "loop is not closed}
lead to stop the bundle adjustment before the convergence . This is illustrated in figure~\ref{ImOldNewTapas}.
 To change this fact, the comportment of Tapas/NewTapas has evolved this way :

\begin{itemize}
    \item  the initial default value of viscosity  on center and rotation has been divided by $10$;
    \item  conversely the constraint to solve the arbitrary ambiguity has bee reactivated (i.e the first image is
           frozen, and the length of the base between two first images is frozen)
    \item  a small initial viscosity has been added on internal parameters;
    \item  in the third {\tt <EtapeCompensation>} the viscosity is continuously decreasing;
    \item  in the fourth step of  {\tt <EtapeCompensation>} the viscosity is highly strongly decreasing and
           the bundle does not stop before a test of convergence is satisfied (the test verify that with an
           accuracy of $1E-10$ the $3d$ point are identical after two consecutive steps).
\end{itemize}

As this new version can increase significantly the computation, the option {\tt RefineAll=false}
allow to limit the number of iteration (by a factor around $2$), of course it has a risk of non convergence \dots


\begin{figure}
\begin{center}
   \includegraphics[width=120mm]{FIGS/NewTapas/Old00.jpg}
   \includegraphics[width=120mm]{FIGS/NewTapas/New00.jpg}
\end{center}
\caption{Result with old and new version of Tapas, with first convergence is visibly not achieved}
\label{ImOldNewTapas}
\end{figure}


        % - - - - - - - - - - - - - - - - - - - - - - - - - - - - - - - - - - - - - - -

\subsection{Additional distortion}

The kernel of Apero has two option that can be useful when attempting to modelize finely camera distortion:

\begin{itemize}
   \item model with high degree of freedom
   \item possibility to define a distortion as composition of several distortion (as described in~\ref{ComposDist}).
\end{itemize}

These option are now partly accessible in Tapas.

There is two family of distortion with high degree of freedom :


\begin{itemize}
   \item high radial distortion made from a polynomial radial distortion and $6$ parameters for degree $2$
         general polynoms; these polynoms are available in Tapas via the {\tt Four7x2, Four11x2, Four15x2, Four19x2}
         and in Apero with {\tt eModeleRadFour7x2, ..., eModeleRadFour19x2}, the {\tt Four19x2} modelize the radial
         distortion with a polynom $r_3 \rho^3 +\dots + r_{19} \rho^{19}$; also I am not sure it maybe necessary to
         use until $ \rho^{19}$, I observe that with modern camera using sophisticated aspherical lenses, it may be
         insufficient to use the so called $r_3 r_5 r_7$ model;


   \item general polynomial model, i.e $D_N(X,Y) \sum (A_{i,j} x^iy^j,B_{i,j} x^iy^j)$  where $i+j\leq N$,
         there accessible in {\tt Apero} a a unified model {\tt eModelePolyDeg2, \dots eModelePolyDeg7},
         for example,{\tt eModelePolyDeg7} has $66$ parameters, because $66=8*9-6$, the $-6$ comes because
         there $6$ polynoms already modelized by focal, principal point and rotations;
\end{itemize}

The first one is accessible directly in {\tt Tapas}, not the second. But both are accessible as additional distortion.
In fact, it would be generally a bad idea to try to estimate directly the $66$ parameters of a {\tt eModelePolyDeg7}, it's
preferable to estimate first a model with physical meaning and few parameters, and then to estimate the high degree polynom as a modification to this physical model. This can be done with :

\begin{itemize}
    \item  {\tt AddFour7x2,  \dots AddFour15x2} for high degree radial models;
    \item  {\tt AddPolyDeg0, \dots AddPolyDeg7} for high degree general models;
\end{itemize}

A possible example of use :

\begin{verbatim}
"mm3d" "NewTapas" "Four15x2" "R.*JPG" "DegGen=2"
"mm3d" "NewTapas" "AddPolyDeg7" "R.*JPG" "InOri=Ori-Four15x2/"
\end{verbatim}

The first call is classical, just remark the {\tt DegGen=2} because by default only $1$ degree
general parameter od {\tt Four15x2} is free.  In the second call we start from the first orientation/calibration
and add a $7$ degree general polynom. Note that, as with {\tt AutoCal} et {\tt Figee} all the calibration must have
an initial value when using the additional mode.

Also note that only the additional distortion will be optimized (else the problem would be far over parameterized).
The result in {\tt Ori-AddPolyDeg7/AutoCal60.xml} looks like :

{\tiny
\begin{verbatim}
<ExportAPERO>
     <CalibrationInternConique>
          <KnownConv>eConvApero_DistM2C</KnownConv>
          <PP>389.315403456829813 291.190537211471565</PP>
          <F>653.34645310282724</F>
          <SzIm>800 600</SzIm>
          <CalibDistortion>
               <ModUnif>
                    <TypeModele>eModeleRadFour15x2</TypeModele>
                    <Params>0.000731887299586956555</Params>
      .....
                    <Etats>499.999999999999943</Etats>
                    <Etats>399.999999999999943</Etats>
                    <Etats>300</Etats>
               </ModUnif>
          </CalibDistortion>
          <CalibDistortion>
               <ModUnif>
                    <TypeModele>eModelePolyDeg7</TypeModele>
                    <Params>-0.00116060037632004101</Params>
                    <Params>0.000376819315728853894</Params>
      .....
                    <Params>0.155804133319492222</Params>
                    <Etats>652.861598677042821</Etats>
                    <Etats>391.129060939043825</Etats>
                    <Etats>287.689359319814059</Etats>
               </ModUnif>
          </CalibDistortion>
     </CalibrationInternConique>
</ExportAPERO>

\end{verbatim}
}

        % - - - - - - - - - - - - - - - - - - - - - - - - - - - - - - - - - - - - - - -

\subsection{Non Linear Bascule (swing)}

\subsubsection{Motivation}

\label{NonLin:GCPBascule}

The {\tt GCPBascule} tool, described in~\ref{Sec:GCPBascule}, transforms a relative orientation
into an absolute one, using at least $3$ ground control point (GCP). The default use make
the assumption that the relative orientation is "perfect" and it computes the minimum number
of parameters, i.e. the seven parameter corresponding to the arbitrary $3d$ similitude
 that can be computed from tie points :

\begin{itemize}
   \item $3$ parameter for rotation;
   \item $3$ parameter for translation;
   \item $1$ parameter for scale.
\end{itemize}


When {\tt MicMac} tools is used for metrology, this geo-referencing by {\tt GCPBascule}
can be insufficient because there appears non linear distortion in the relative orientation. The discussion about the
origin and the quantification of these effect,  is quite complex and cannot be discussed here;
however,  to solve it we have to input more GCP that the minimum required and to use these
surplus GCP for correcting this distortion. In {\tt MicMac} tools, there is two way to do it :


\begin{itemize}
   \item the "classical" way is to do a compensation with high weighting on the GCP, this can be done
        with the simplified tool   {\tt Campari}  (\ref{CAMPARI} and \ref{Bundle:CAMPARI});

   \item a non standard way is to use the redundancy of the GCP to directly estimate the non linear distortion
        existing between the result of "Bascule" (swing) ans the ground truth; this is what is described here;

\end{itemize}

At the time of writing these documentation, it's not clear which method is "better". The first one is more
standard and seems more correct from theoretical point of view, however from experimental point of view, the
second one seems more accurate.


\subsubsection{Mathematical model}

Let :

\begin{itemize}
   \item let $G_k$ be the ground coordinate of the GCP;
   \item let $I_k$ be the coordinate of the GCP in relative initial model (estimate by bundle intersection);
   \item let $T$ be the transformation from relative to absolute we want to estimate similitude, such that  $G_k \approx T(I_k)$ ;
   \item let $S$ be the initial similitude estimation of $T$;
   \item let $C$ be the "small" correction we want to do compute $T= (Id+C) \circ S$
\end{itemize}


Also, it is current that the acquisition is "linear" and that the coordinate must not be treated
symmetrically. For example if the acquisition is made from a single strip, let $O'X'Y'Z$ be a
coordinate system ,  centered on the acquisition, such that the image center are aligned on the $O'X'$ axes.
Concretely $O'X'Y'Z$ can be estimated automatically from initial $OXYZ$ by elementary computation of inertial axes of the
image center (i.e. computation of order $2$ moments) \footnote{this is exactly what is done in MicMac}.



Basically, the model for estimate $C$ is restricted  quadratic function on of  each coordinates $X'Y'Z$:

\begin{itemize}
   \item $C(X',Y',Z') = (X'^c,Y'^c,Z'^c)$;
   \item $X'^c = \sum c^x_{ij} X'^i Y'^j$
   \item $Y'^c = \sum c^y_{ij} X'^i Y'^j$
   \item $Z'^c = \sum c^z_{ij} X'^i Y'^j$
\end{itemize}


For example, suppose that the acquisition is made from a single strip, and
that the errors is only on $Z^c$,  and that this error depends only on  the
"main" variable $X'$, a possible mode could be :

\begin{itemize}
   \item $X'^c = 0$
   \item $Y'^c = 0$
   \item $Z'^c =  c^z_{00} + c^z_{10} X' + c^z_{20} X'^2$
\end{itemize}

Also if we have a sufficient number of tie points \footnote{$6$ is the minimum, but $12$ would be more
reasonable} and have high distortion we can select the full model.


\subsubsection{Using it in MicMac}

The command {\tt GCPBascule} has several parameters to compute a non linear
correction. Of course by default, the swing-bascule if made with the standard $7$ parameters
with no correction. The non linear correction is activated if the optional {\tt PatNLD} is
used. The meaning of the different parameters is then :


\begin{itemize}
  \item  {\tt PatNLD} :  define the pattern of the GGP name that will be used to estimate  $C(X',Y',Z')$;
         in the final application probably one will use {\tt "PatNLD=.*"} to specify that all GCP; however,
         if one want to estimate the accuracy with GCP that are not used for the estimation, it can be
         convenient to specify a subset:
   \item  {\tt NLDegX}, {\tt NLDegY} and  {\tt NLDegZ} specify the monoms that be used for  $X'^c,Y'^c,Z'^c$,
          it a vector of strings  which elements must belongs to {\tt \{1,X,Y,X2,XY,Y2\}};
   \item  {\tt NLFR}  : as the function $T$ is no longer a pure similitude, the orientation of each image
          can no longer be exactly a rotation matrix; the parameter {\tt NLFR} mean "Non Linear For Rotation" and
          control the export is done ; if :

\begin{itemize}
     \item   {\tt NLFR} is false, {\tt MicMac} will export for each image the closest matrix that fit
             with new system, if $X'^c,Y'^c,Z'^c$ contains non linear term there will be some error but
             probably very small; the matrix will be non rotation matrix which mean that there will be
             no longer usable compensation (but usable in matching);

     \item   {\tt NLFR} is true, {\tt MicMac} will export for each image the closest rotation that fit
             the new system; of course the price to pay for having true rotation is that the error will be bigger;

\end{itemize}

   \item  {\tt NLShow}  : give detailed information;
\end{itemize}




\subsubsection{Example of use, and message interpretation}

Here is a possible use :

{\small
\begin{verbatim}
 "mm3d" "GCPBascule" ".*ARW" "Ori-AllRell-F15AddP7/" "Basc-Def-NonO" "GCP.xml" "MesFinal-S2D.xml" \
   "PatNLD=(3|7|14).*" "NLDegZ=[1,X,X2]" "NLDegX=[1,X,Y]" "NLDegY=[1,X,Y]" "NLFR=false" "NLShow=true"
\end{verbatim}
}

The message should look like that , first classical message of {\tt GCPBascule} :

\begin{verbatim}
BEGIN Pre-compile
NEW CALIB TheKeyCalib_350
NB[10a]= 11
  .....
NB[9c]= 11
BEGIN Load Observation
Pack Obs NKS-Set-Orient@-AllRell-F15AddP7 NB 180
BEGIN Init Inconnues
NUM 0 FOR 001aDSC01061.ARW
  .....
NUM 179 FOR 242bDSC01005.ARW
BEGIN Compensation
BEGIN AMD
END AMD
\end{verbatim}

Then  a message remembering the monom used for $X'^c,Y'^c,Z'^c$ :

\begin{verbatim}
 MQ:X [1 X Y ]
 MQ:Y [1 X Y ]
 MQ:Z [1 X X2 ]
\end{verbatim}

Then the error before and after non linear correction :

\begin{verbatim}
* 14a ErInit : 0.0723239 => ErCor : 0.0260313 DZ=-0.0254591
* 14b ErInit : 0.043778 => ErCor : 0.0135564 DZ=0.00440725
* 14c ErInit : 0.0277668 => ErCor : 0.0253775 DZ=0.0211511
  13a ErInit : 0.0583578 => ErCor : 0.0456215 DZ=-0.042776
  13b ErInit : 0.0361378 => ErCor : 0.0231452 DZ=-0.0176777
  13c ErInit : 0.0200873 => ErCor : 0.020112 DZ=0.00742163
   ....
\end{verbatim}

Let detail the message for point {\tt  14a} :
\begin{itemize}

 \item {\tt * 14a ErInit : 0.0723239 => ErCor : 0.0260313 DZ=-0.0254591};
 \item the {\tt *} means that point {\tt  14a} belongs to {\tt PatNLD}
 \item {\tt 0.0723239} is the initial distance $||G_k - S(I_k)||$ (after bascule-swing);
 \item {\tt 0.0260313} is the distance after non linear correction $||G_k - T(I_k)||$ ;
 \item {\tt -0.0254591} is the $Z$ value of $||G_k - T(I_k)||$ (generally the most important);

\end{itemize}


        % - - - - - - - - - - - - - - - - - - - - - - - - - - - - - - - - - - - - - - -




        % - - - - - - - - - - - - - - - - - - - - - - - - - - - - - - - - - - - - - - -

\subsection{A detailed example}

        % - - - - - - - - - - - - - - - - - - - - - - - - - - - - - - - - - - - - - - -
\subsection{Miscellaneous options to Tapas}

\subsubsection{FreeCalibInit}

Impose that all calibration parameters are freed at the begining. Rarely usefull \dots 

\subsubsection{FrozenCalibs}

Same role as {\tt FrozenPoses} but for internal calibration.



\subsubsection{SinglePos}

A pattern (generaly one) of pose and calibs in the form {\tt [PatPose,PatCalib]}. When specified :

\begin{itemize}
   \item the RefineAll is set to false;
   \item only this pose and calbration will be saved;
\end{itemize}



%=======================================================================================================
\section{GCP : accuracy and optimal weighting}

When using GCP mixed with tie points may rise the following difficulties :


\begin{itemize}

 \item what is the accuracy of the geo-referencing ?
 \item What is the optimal weighting of the GCP ?

\end{itemize}

Obviously, the same GCP can be used in the optmisation process and in the accuracy measurement, else
the result would be obviously biased and the "optimal" weighting would be $\infty$.  The safe alternative
is to separate the GPC used for optimisation and those used for measuring accuracy. When one has
as many GCP as wanted this work perfectly well, but the the problems appear when there is few GCP.

For a given weighting, the classical alternative is to proceed for estimating accuracy is to proceed this way :

\begin{itemize}

 \item parse the GCP and alternatively each GCP is considerer ejected;
 \item run the bundle without the ejected GCP and memorize the accuracy on this ejected GGP;
 \item the global accuracy of you bundle can be estimated as the average of the accuracy of the ejected 
       GCP.
\end{itemize}

The question remain of what is the optimal weighting ? Also there is many theoreticall consideration,
my personnal opinion is that the safest way is to do it purely empirically by testing different
weighting with the previous method (which I consider as safe).  Of course the cost to pay, is
computation time \dots

This testing can be done by the {\tt MulRTA} option of {\tt Campari}. When used,  {\tt MulRTA}
is a vector of double it used this way ,
 let $P_T$ and $P_I$ be the ground and image accuracy (set by  secong and fourth value of
optionnal parameter {\tt GCP}) , for each value $M$ of $MulRTA$

\begin{itemize}
       \item set the Ground and image accuracy to $M*P_T$ and  $M*P_I$
       \item for each GGP $G$ :
       \begin{itemize}
             \item run the bundle without using $G$ in the measurement;
             \item compute the distance ${D^M}_G$ between ground measurment and bundel measurment;
       \end{itemize}
       \item memorise the value $Ac(M)=\sum_G {D^M}_G$ as an estimator of the accuracy with $M$ weighting.
\end{itemize}

The results are stored in a file {\tt SauvRTA.xml}. For example :

\begin{verbatim}
<XmlResultRTA>
     <BestMult>0.3 </BestMult>
     <BestMoyErr>0.036148</BestMoyErr>
     <RTA>
          <Mult>3</Mult>
          <MoyErr>0.07951</MoyErr>
          <OneAppui>
               <Name>P123</Name>
               <EcartFaiscTerrain>-0.1311860 -0.073748 -0.03087</EcartFaiscTerrain>
               <DistFaiscTerrain>0.1536285</DistFaiscTerrain>
               <EcartImMoy>0.473698</EcartImMoy>
               <EcartImMax>0.79826</EcartImMax>
               <NameImMax>OIS-023.tif</NameImMax>
          </OneAppui>
   ....
     </RTA>
     <RTA>
          <Mult>1</Mult>
          <MoyErr>0.04443</MoyErr>
   ....
     </RTA>
   ....
<XmlResultRTA>
\end{verbatim}

Also the interpretation should be quite obvious, a brief comment :

\begin{itemize}
   \item  the best accuracy was reached for $M=0.3$, its value was $0.036148$;

   \item for  $M=3$ , the estimated accuracy estimated was $0.07951$;

   \item For $M=3$ , and  $G={\tt P123}$, some detail are memorized :
   \begin{itemize}
       \item the  difference bewteen bundle estimation and ground is in {\tt <EcartFaiscTerrain>}
       \item the  norm of  {\tt <EcartFaiscTerrain>} is {\tt <DistFaiscTerrain>}
       \item the   image accuracy  is  {\tt  <EcartImMoy>} ;
       \item the   worst image seizing of {\tt P123} was done on image {\tt OIS-023.tif}.
   \end{itemize}
  
\end{itemize}

%=======================================================================================================

\section{Initial Orientation with Martini}

\label{Ori:Martini}

The command {\tt Martini} \footnote{MARTingale d'INItialisation} computes
 initial values of orientation while
aiming to solve some ressource issues of {\tt Tapas} on memory and computation time. The principle 
of {\tt Martini} is :

\begin{itemize}
   \item  compute the relative orientation of pairs and triplet;
   \item  build a global orientation coherent with the constraints given
          by pairs and triplets;
\end{itemize}

Althouh the second part is still not fully satisfying, {\tt Martini} can 
already be used now as it solve sometimes some
orientation problem that were not solved correctly by Tapas 
(and generally faster).  Also preliminar execution of  {\tt Martini} 
is necessary for some commands as {\tt OriRedTieP}

\begin{verbatim}
mm3d Martini
*****************************
*  Help for Elise Arg main  *
*****************************
Mandatory unnamed args : 
  * string :: {Image Pat}
Named args : 
  * [Name=OriCalib] string :: {Orientation for calibration }
  * [Name=Exe] bool :: {Execute commands, def=true (if false, only print)}
  * [Name=SH] string :: {Prefix Homologue , Def=""}
\end{verbatim}


The signification of parameters 

\begin{itemize}
   \item  first one : standard pattern of images to orientate;
   \item  {\tt OriCalib} when given specify the folders were internal calibration can be found; as {\tt Martini} will
          not do any adjustment of internal calibration, it is highly recommanded to use this option with a 
          relatively good internal calibration;
   \item  {\tt SH} if Martini must be used with a folder of homologous point different from the standard {\tt Homol/}.
\end{itemize}

The result of martini are stored in an orientation folder {\tt Ori-Martini/} when no {\tt OriCalib]} was set and
,for example, {\tt Ori-MartiniTOTO/} if Martini was used with {\tt OriCalib=TOTO}.

%=======================================================================================================

\section{Miscellaneous  tools about calibration}

  % - - - - - - - - - - - - - - - - - - - - -

\subsection{{\tt ConvertCalib} to Calibration conversion}

Sometimes it may be necessary to export a given calibration from one model to another model. For example from Fraser terrestrial
model to aerial model. Of course as the mathematicall modelisation of the camera is not the same, this conversion will 
generally imlply some lost of accuracy.  The tool {\tt ConvertCalib} allow to do such conversion.

\begin{verbatim}
mm3d ConvertCalib
*****************************
*  Help for Elise Arg main  *
*****************************
Mandatory unnamed args : 
  * string :: {Input Calibration}
  * string :: {Output calibration}
Named args : 
  * [Name=NbXY] INT :: {Number of point of the Grid}
  * [Name=NbProf] INT :: {Number of depth}
  * [Name=DRMax] INT :: {Max degree of radial dist (def=depend Output calibration)}
  * [Name=DegGen] INT :: {Max degree of generik polynom (def=depend Output calibration)}
  * [Name=PPFree] bool :: {Principal point free (Def=true)}
  * [Name=CDFree] bool :: {Distorsion center free (def=true)}
  * [Name=FocFree] bool :: {Focal free (def=true)}
  * [Name=DecFree] bool :: {Decentrik free (def=true when appliable)}
\end{verbatim}


The first argument is a file containg the calibration to be converted. The second is a file that contain a model of the
targeted calibration. The folder {\tt Documentation/Data/DocConvertCalib} contains data to test this tool :

\begin{itemize}
   \item  {\tt Aerial.xml} a calibration with aerial model ({\tt PPA} and {\tt PPS} separated);
   \item  {\tt Fraser-Affine.xml} a calibration with fraser model ({\tt PPA} and {\tt PPS} merged, decentric distorsion);
\end{itemize}

For example , if we want to produce a convertion to fraser model, without affine distorsion and a one parameter of radial
distorsion, we can run :

\begin{verbatim}
mm3d ConvertCalib Aerial.xml Fraser-Affine.xml  DRMax=1 DegGen=0
....
   ============================= ERRROR MAX PTS FL ======================
   ||    Value=0.145258 for Cam=Fraser-Affine.xml and Pt=Pt_0_0_1 ; MoyErr=0.0623394
   ======================================================================

--- End Iter 16 ETAPE 0

\end{verbatim}


The average accuracy will be $0.063$ pixel, it is measured as the average error re-projection of  synthetic $3d$ points.

  % - - - - - - - - - - - - - - - - - - - - -

\subsection{{\tt ConvertOriCalib} to Calibration conversion}

Sometime it is usefull to convert orientations  using a new internal
calibration. Suppose :

\begin{itemize}
   \item user  has computed orientation of $1000$ images with internal cabration $Cal_1$;

   \item finally user woulr prefer to use   a different calibration  $Cal_2$ but do not want to
         recompute all external orientation;

   \item if user just replace  $Cal_1$ by $Cal_2$, the result can be very poor because of correlation
         between internal and external parameters; 

   \item for example, in aerial photogrammetry, if the 
         focal lenght is significativelly different bewteen $Cal_1$ and $Cal_2$,
         the height of position should be change to compensate as much as possible the
         variation of focal;

\end{itemize}

The command {\tt ConvertOriCalib} offers this possibility :

\begin{verbatim}
mm3d ConvertOriCalib -help
*****************************
*  Help for Elise Arg main  *
*****************************
Mandatory unnamed args : 
  * string :: {Pat or file}
  * string :: {Input global orientation}
  * string :: {Targeted   internal orientatin}
  * string :: {Ouptut prientation folder}
Named args : 
\end{verbatim}

An example with a data set :

\begin{verbatim}
mm3d ConvertOriCalib Stereo-MG_00.*CR2 Ori-BascF15/ Ori-Peturb/ ConverOC
\end{verbatim}

Here, the resulting orientation is saved under folder {\tt Ori-ConverOC/}.




  % - - - - - - - - - - - - - - - - - - - - -

\subsection{{\tt Genepi} to generate articficial perfect 2D-3D points}

It generate a set of $3D$ points and their images projection for a given camera.

This tools was used for internal checking. Not sure it will be used very often, maybe sometime to export 
MicMac's orientation in other format when no better solution is avalaible or also usefull when preparing
data sets for students.

\begin{verbatim}
   mm3d Genepi _MG_008.*.CR2 Ori-AllFix/ -help
*****************************
*  Help for Elise Arg main  *
*****************************
Mandatory unnamed args : 
  * string :: {Image}
  * string :: {Orientation}
Named args : 
  * [Name=NbXY] INT :: {Number of point of the Grid}
  * [Name=NbProf] INT :: {Number of depth}

\end{verbatim}

The first parameters define the  pattern of images and the second the orientation. The optionall parameters control
the density of points.



\subsection{{\tt Init11P}, space resection for uncalibrated camera}

The space resection compute the pose of a camera from a set of $3d$ point and their corresponding
image projection. The {\tt Init11P} deals with the case where the calibration is unknown. In this case the
following parameter are :


\begin {itemize}
    \item center of camera ($3$ parameters);
    \item orientation of camera ($3$ parameters);
    \item focal and principal point ($3$ parameters);
    \item skew and ratio $\frac xy$ ($2$ parameters).
\end {itemize}

At least $6$ projection are required to compute these $11$ parameters. 



\begin{verbatim}
mm3d Init11P
*****************************
*  Help for Elise Arg main  *
*****************************
Mandatory unnamed args : 
  * string :: {Name File for GCP}
  * string :: {Name File for Image Measures}
Named args : 
  * [Name=FM] bool :: {Fraser Mode, use all affine parmeters (def=false)}
  * [Name=Rans] vector<std::string> :: {Parameters for Ransac, [NbTirage,PropInlier]}
  * [Name=Filter] string :: {Filter for Image (Def=.*)}
\end{verbatim}

The mandatory parameters are :

\begin {itemize}
  \item the file for {\tt GCP}, containing a {\tt DicoAppuisFlottant} structure ;
  \item the file for image measurement containing a {\tt SetOfMesureAppuisFlottants} structure.
\end {itemize}

For all the images contained in the image file an orientation is created in the folder {\tt Ori-11Param/}.
If the optional {\tt FM} is set to true, the $11$ parameters are exported as a Fraser camera. If it is
set to false, the skew and xy ratio are forced to $0$ and $1$, and the result is exported as a radial camera (with
no distorsion); however, $6$ are required as forcing skew to $0$ and xy ratio to $1$ is done
\emph{a posteriori}. 

The folder {\tt Ori-11ParamComp/} also contains the result of a compensation done using the previous value.
They may be sligthly different and theoretically more accurate, specifically when the {\tt FM=false} option.

By default {\tt Init11P}  assume that the data contain no gross errori and dont try
to make a robust estimation (the idea is generally the point has been seized by a human operator 
and that they are few but correct). If it is not the case, for example if the point
come from an automatic computation with {\tt Im2XYZ} described in~\ref{Im2XYZ:Hom} .
In this case, it is possible to use a ransac estimation with the optional parameter
which must contain two value :

\begin{itemize}
   \item  the number of sampling;
   \item  an estimation of the proportion of inlier (can be very rough but better to underestimate it);
\end{itemize}



\subsection{{\tt Aspro}, space resection for calibrated camera}


The tool {\tt Aspro} allows to orientate a set of images with known internal calibration
 from existing $3d$ points and their corresponding image projection.

\begin{verbatim}
 mm3d Aspro -help
*****************************
*  Help for Elise Arg main  *
*****************************
Mandatory unnamed args : 
  * string :: {Name File for images}
  * string :: {Name File for input calibration}
  * string :: {Name File for GCP}
  * string :: {Name File for Image Measures}
\end{verbatim}

The parameters should be quite intutive, an example of use :

\begin{verbatim}
mm3d Aspro "_MG_008[0-3].CR2" Ori-AllFix/ TestOPA-S3D.xml TestOPA-S2D.xml 
\end{verbatim}

The resulting orientation are stored in {\tt Ori-Aspro/}. Note that {\tt MicMac} must be abble
to find the internal calibration with name respecting its convention in the folder of calibration
( {\tt Ori-AllFix/}  here). As the naming of {\tt MicMac} may be not obviuous, the tool 
described in~\ref{TestNameCalib} can be useful.

%=======================================================================================================
\section{Rigid Block Compensation}

  % - - - - - - - - - - - - - - - - - - - - -
\subsection{Introduction}

\subsubsection{Mathematics}

These functionalities are used when the camera form a rigid block, which mean that the relative position
of a set of camera do not change during time (or changes are small, whatever it means).

Supose we have a sets of camera $Cam_A,Cam_B,Cam_C \dots $ . Let $P_{A,k}$ be the pose of image acquired by camera $A$
at time $k$ (idem $P_{B,k} \dots$).  The rigidity hypothesis means that :


\begin{equation}
     \forall k,k',\alpha,\beta  : \;
     P_{\alpha,k}^{-1} P_{\beta,k} =  P_{\alpha,k'}^{-1} P_{\beta,k'}
     \label{EQ:RIG1}
\end{equation}

If we decompose the pose  $P_{A,k} = (C_{A,k},R_{A,k})$ , $C_{A,k}$ being the center 
and $R_{A,k}$ the rotation matrix, equation \ref{EQ:RIG1}  writes :

\begin{equation}
     0 = R_{\alpha,k}^{-1} R_{\beta,k} -  R_{\alpha,k'}^{-1} R_{\beta,k'}
      = \Delta_{\omega}(\alpha,\beta,k,k')
     \label{EQ:RIG2}
\end{equation}

\begin{equation}
     0 = (- C_{\alpha,k}  + R_{\alpha,k}^{-1} * C_{\beta,k}) -(- C_{\alpha,k'}  + R_{\alpha,k'}^{-1} * C_{\beta,k'})
      = \Delta_{Tr}(\alpha,\beta,k,k')
     \label{EQ:RIG3}
\end{equation}

Sometime it may be convenient to introduce the, may it be known or unknown, 
the global pose calibration of the block $Q_A,Q_B \dots$; the
equations \ref{EQ:RIG2}, \ref{EQ:RIG3} become  with  $Q_A=(c_A,r_A), Q_B= \dots $:

\begin{equation}
     0 = R_{\alpha,k}^{-1} R_{\beta,k} -  r_{\alpha}^{-1} r_{\beta}
      = \Delta^g_{\omega}(\alpha,\beta,k)
     \label{EQ:RIG4}
\end{equation}

\begin{equation}
     0 = (- C_{\alpha,k}  + R_{\alpha,k}^{-1} * C_{\beta,k}) -(- c_{\alpha}  + r_{\alpha}^{-1} * c_{\beta})
      = \Delta^g_{Tr}(\alpha,\beta,k)
     \label{EQ:RIG5}
\end{equation}


\subsubsection{Data set}

\begin{figure}
\begin{center}
   \includegraphics[width=60mm]{FIGS/Rigid-Block/Fuji.jpg}
\end{center}
\caption{The amateur Fuji stereo camera used for  creating the rigid data set}
\label{ImFuji}
\end{figure}

The data set to illustrate are located in the folder {\tt Documentation/Data/Rigid-Block} it was acquired using
a Fuji stereo camera as the one shown on figure~\ref{ImFuji}. The data set contains :

\begin{itemize}
    \item three files to {\tt MPO}  format, this the format used by Fuji to store pair of images of the stereo camera;
    \item the two files containing measurement of GCP ({\tt AllGCP-RTL.xml} and {\tt MesuresFinales-S2D.xml})
    \item the folder {\tt Ori-Calib/} containing calibration of the images; in fact to limit the size, the dataset
          is limited to three images and it would not have been possible to do any valuable self calibration;

    \item the "classical" file {\tt MicMac-LocalChantierDescripteur.xml} which contain will be detailled later;
    \item a file {\tt Cmd.txt} containg the command that can be bacthed.
\end{itemize}


\subsubsection{Preprocessing and camera naming}

Each {\tt MPO} files contains $2$ jpg images, the first  command extract these images : 

\begin{verbatim}
mm3d SplitMPO DSCF329.*MPO

ls DSCF329*jpg
DSCF3297_L.jpg	DSCF3298_L.jpg	DSCF3299_L.jpg
DSCF3297_R.jpg	DSCF3298_R.jpg	DSCF3299_R.jpg
\end{verbatim}

All the images {\tt DSCF\dots\_L.jpg} correspond the left camera and {\tt DSCF\dots\_R.jpg} to the right.
As left and right images correspond to different  camera, it is important that {\tt MicMac} create
and recognize different calibration files. By default, the images having the same focal and same camera
model indicated in the {\tt xif}, there may be some conflict. To avoid this, it is possible to define
a user specific identifier that will be added to the camera name, this is done by redifing the key
{\tt NKS-Assoc-StdIdAdditionnelCam}:

\begin{verbatim}
 <KeyedNamesAssociations>
        <Calcs>
            <Arrite> 1 1 </Arrite>
            <Direct>
                <PatternTransform>  DSCF([0-9]{4})_(.)\.jpg </PatternTransform>
                <CalcName> Fuji-$2 </CalcName>
                </Direct>
        </Calcs>
        <Key> NKS-Assoc-StdIdAdditionnelCam </Key>
    </KeyedNamesAssociations>
\end{verbatim}

With this key, we assure that right and left camera will have a different identifier (and also, conversely, that 
this identifier do not depend  of image number). We can test this by using the command  {\tt TestNameCalib} :


\begin{verbatim}
mm3d TestNameCalib DSCF3297_L.jpg
./Ori-TestNameCalib/AutoCal_Foc-6300_Cam-FinePix_REAL_3D_W1Fuji-L.xml

mm3d TestNameCalib DSCF3297_R.jpg
./Ori-TestNameCalib/AutoCal_Foc-6300_Cam-FinePix_REAL_3D_W1Fuji-R.xml
\end{verbatim}

\subsubsection{Standard MicMac Processing}

The first three command are standard micmac processing :

\begin{verbatim}
Tapioca All D.*jpg 1500

Tapas Figee D.*jpg InCal=Calib Out=AllRel

GCPBascule D.*jpg  AllRel Basc AllGCP-RTL.xml MesuresFinales-S2D.xml
\end{verbatim}

As usual :

\begin{itemize}
    \item {\tt Tapioca} to compute tie point;
    \item {\tt Tapas} to compute the relative orientation, as said before with only $3$ images we use an existing internal
          calibration ({\tt InCal=Calib}) and maintain it frozen to its initial value ({\tt Figee});
    \item then we transfer to an absolute geo reference system with {\tt GCPBascule};
\end{itemize}

  % - - - - - - - - - - - - - - - - - - - - -
\subsection{Indicating block structure}

To understand the block structure and use equations like  \ref{EQ:RIG1} \dots \ref{EQ:RIG5}, 
{\tt MicMac} will need to compute from the name of images to which camera it belongs to and 
what were the images aquired during the same time. This is done by creating a single reversible key
that will return $2$ values, here :

\begin{verbatim}
   <KeyedNamesAssociations>
        <IsParametrized>  true </IsParametrized>
        <Calcs>
            <Arrite>  2 1 </Arrite>
             <Direct>
                <PatternTransform> DSCF([0-9]{4})_(.)\.jpg   </PatternTransform>
                <CalcName> $1  </CalcName>
                <CalcName> $2  </CalcName>
             </Direct>
             <Inverse>
                <PatternTransform> (.*)%(.*) </PatternTransform>
                <CalcName> DSCF$1_$2.jpg  </CalcName>
                <Separateur > % </Separateur>
             </Inverse>
        </Calcs>
        <Key>  Loc-Assoc-Im2Block </Key>
    </KeyedNamesAssociations>
\end{verbatim}

The first values correspond to $k,k'$ of equation   \ref{EQ:RIG1} \dots while the second correspond
to the $A,B \dots$.  Also it shoul be pretty obvious, here are some example of result of this key :

\begin{itemize}
    \item {\tt DSCF3297\_R.jpg}  $\Rightarrow$   {\tt 3297}  $\times$    {R};
    \item {\tt DSCF3297\_L.jpg}  $\Rightarrow$   {\tt 3297}  $\times$    {L};
    \item {\tt DSCF3298\_R.jpg}  $\Rightarrow$   {\tt 3298}  $\times$    {R};
    \item \dots
\end{itemize}


  % - - - - - - - - - - - - - - - - - - - - -

\subsection{Block estimation}\label{block_estimation}

The first step is to estimate the, supposed fixe, position of camera relatively to each others. For this
we need to estimate the pose  $Q_A,Q_B \dots$ . As obviously,  everything is undeterminade up to a global
roto-translation, we have to fix an arbirtray constraint, for example $Q_A=Id$, where $A$ is the first
camera ({\tt MicMac} use alphabetic order to select this arbitray reference).  Then with this constraint,
using equation \ref{EQ:RIG4}  and \ref{EQ:RIG5} , for any camera $\beta$, knowing the pose at one time 
$k$ is sufficient to estimate  $r_{\beta}$ and $c_{\beta}$. Generally we have several $k$, and we can
estimate a more accurate valure by simply averaging the estimation.

The command {\tt Blinis} does this computation :

\begin{verbatim}
mm3d Blinis 
*****************************
*  Help for Elise Arg main  *
*****************************
Mandatory unnamed args : 
  * string :: {Full name (Dir+Pat)}
  * string :: {Orientation in}
  * string :: {Key for computing bloc structure}
  * string :: {File for destination}
\end{verbatim}


The meaning shoud be quite obvious, and an example of use :

\begin{verbatim}
mm3d Blinis DSCF329.*jpg Ori-Basc/ Loc-Assoc-Im2Block Blinis.xml

....
=================================================
  EstimCurOri DSCF3297_L.jpg DSCF3297_L.jpg
    [0,0,0] -6.93889e-18 6.67869e-17 9.26854e-34
  EstimCurOri DSCF3298_L.jpg DSCF3298_L.jpg
    [0,0,0] 9.54098e-18 -1.06685e-16 3.19245e-33
  EstimCurOri DSCF3299_L.jpg DSCF3299_L.jpg
    [0,0,0] 1.38778e-17 -2.77556e-17 -5.55112e-17
  ==========  AVERAGE =========== 
    [0,0,0] tetas 5.49329e-18  -2.25514e-17  -1.85037e-17
    DispTr=0 DispMat=6.21513e-17
=================================================
  EstimCurOri DSCF3297_L.jpg DSCF3297_R.jpg
    [0.0772782,0.00105963,4.59993e-05] 0.00587422 0.041968 0.00953594
  EstimCurOri DSCF3298_L.jpg DSCF3298_R.jpg
    [0.0775943,0.00110819,0.000728747] 0.00590563 0.04163 0.00881231
  EstimCurOri DSCF3299_L.jpg DSCF3299_R.jpg
    [0.0775121,0.00141723,-0.00103102] 0.00589855 0.0417358 0.00904956
  ==========  AVERAGE =========== 
    [0.0774615,0.00119502,-8.54263e-05] tetas 0.00589282  0.041778  0.0091326
    DispTr=0.000688418 DispMat=0.000271402
--- End Iter 1 ETAPE 0
\end{verbatim}

{\tt MicMac} print for each time $k$ and each camera $\beta$ the value estimated of 
$r_\beta$ and $c_\beta$.  These value will be rarely usefull, also we can observe that
for the first camera we have almost $c_A=[0,0,0]$ and $r_A=Id$ (the angle are 
in fact printed). The average value is also computed. In fact the only value printed
of real insterest will generally be the dispertion :  {\tt  DispTr} and {\tt DispMat}.

The interesting result of this command rely in the file created , here {\tt Blinis.xml}.
In contains in a {\tt xml} specification, that should be pretty obvious, the value
of the estimated block calibration.
In our example the file contains :


\begin{verbatim}
<StructBlockCam>
     <KeyIm2TimeCam>Loc-Assoc-Im2Block</KeyIm2TimeCam>
     <LiaisonsSHC>
          <ParamOrientSHC>
               <IdGrp>L</IdGrp>
               <Vecteur>0 0 0</Vecteur>
               <Rot>
                    <L1>1 -5.49329100725988965e-18 2.25514051876984922e-17</L1>
                    <L2>5.49329100725988888e-18 1 1.85037170770859382e-17</L2>
                    <L3>-2.25514051876984922e-17 -1.85037170770859382e-17 1</L3>
                    <TrueRot>true</TrueRot>
               </Rot>
          </ParamOrientSHC>
          <ParamOrientSHC>
               <IdGrp>R</IdGrp>
               <Vecteur>0.0774615118759180987 0.00119501688228340619 -8.54262594655412071e-05</Vecteur>
               <Rot>
                    <L1>0.999110080792062982 -0.00627395825477376507 -0.0417095181882368854</L1>
                    <L2>0.00588764369540181742 0.999938688556114119 -0.0093784209968844328</L2>
                    <L3>0.0417658007392831265 0.00912450417809996389 0.999085762741172445</L3>
                    <TrueRot>true</TrueRot>
               </Rot>
          </ParamOrientSHC>
     </LiaisonsSHC>
</StructBlockCam>

\end{verbatim}

These value are coherent with the specification of the fuji camera; the $L$ camera being arbitrary
set in identity position, we  analyse the $R$ camera :

\begin{itemize}
    \item the base is parallel to the $X$ axes:
    \item the lenght of the base is $7.7 cm$ (approximatively the spacing between eyes of adult human);
    \item the two camera are approximatively parralel, with a slight convergence such that the maximal overlap
          occurs at $2 meters$.
\end{itemize}


  % - - - - - - - - - - - - - - - - - - - - -

\subsection{Block compensation}

The {\tt Blinis} command does an \emph{a posteriori} estimation of the rigid calibration,
that is usefull to compute a reasonnable initial value, but it does not do any compensation.
The compensation can be do in {\tt Campari} using several optional parameters.

\subsubsection{Global with no attachment to known value }

The first case correspond to the following hypothesis :

\begin{itemize}
    \item each block of camera is closed to the same value $Q_A, Q_B \dots $;
    \item we don't have any good estimation of these value;
    \item how close the camera are equals to these common value is given by the a priori
          variance that we have to indicate  $\sigma^g_{Tr} $ and $\sigma^g_{\omega} $
\end{itemize}

Then we will add to global minimization of the bundle, the term $E(\sigma^g_{Tr},\sigma^g_{\omega})$
given by the following equation :

\begin{equation}
      E^g(\sigma^g_{Tr},\sigma^g_{\omega}) =
      \sum _{k,\beta}
      (\frac{\Delta^g_{\omega}(A,\beta,k)}{\sigma^g_{\omega}}) ^2
      +(\frac{\Delta^g_{Tr}(A,\beta,k)}{\sigma^g_{Tr}})^2
\end{equation}

Note that the term is  dissymetric as the first camera play a special role. 
This may evolve in future versions (at least optionnaly).
The option  {\tt BlocGlob} of Campari allow to add such term to the bundle. 

\begin{verbatim}
mm3d Campari -help
...
  * [Name=BlocGlob] vector<std::string> :: {Param for Glob bloc compute [File,SigmaCenter,SigmaRot,?MulFinal,?Export]}
  * [Name=OptBlocG] vector<std::string> :: {[SigmaTr,SigmaRot]}
  * [Name=BlocTimeRel] vector<std::string> :: {Param for Time Reliative bloc compute [File,SigmaCenter,SigmaRot,?MulFinal,?Export]}

\end{verbatim}

The parameter {\tt BlocGlob} is a vector that  contain $3$ mandatory values  and $2$ optionnal :

\begin{itemize}
    \item parameter $P_1$ is the name of calibration as created by the {\tt Blinis} command (or also by 
          {\tt Campari});
    \item parameter $P_2$ and $P_3$ correspond respectively to  $\sigma_{Tr}$ and $\sigma_{\omega}$
    \item  parameter $P_4$ will be explained just after (default value is $1.0$);
    \item  parameter $P_5$ is the file containing the output of new estimaded value of block calibration,
           its defaut value is {\tt Out-}$P_1$;
\end{itemize}

The role of the parameter $P_4$ is to allow the  $\sigma_{Tr}$ and $\sigma_{\omega}$ to evolve during
the different iteration of the bundle. It may happen that we know that "at the end" the rigidity block
is very strict, but as we are not so well initialized, we can impose it strictly at the first iteration.
So the value of $\sigma_{Tr}$ and $\sigma_{\omega}$ will be :

\begin{itemize}
    \item $P_2$ and $P_3$ for the first iteration;
    \item $P_2*P_4$ and $P_3*P_4$ for the last iteration;
    \item in between they will evolve folowing a geometric law.
    \item note that if we want to slow down the evolution we can increase the number of iteration
          with the {\tt NbIterEnd} parameter;
\end{itemize}

Here is an example with the data set:

\begin{verbatim}
Campari "DSCF32.*jpg" Basc Cmp GCP=[AllGCP-RTL.xml,0.3,MesuresFinales-S2D.xml,0.3] BlocGlob=[Blinis.xml,1e-3,1e-2,1e-4]
| |  Residual = 0.721748

....

| |  Residual = 1.0723 ;; Evol, Moy=0.00715547 ,Max=9.87406
...
================================================
  ==========  AVERAGE =========== 
    [0,0,0] tetas 5.49329e-18  -2.25514e-17  -1.85037e-17
    DispTr=2.09346e-16 DispMat=5.38957e-17
=================================================
  ==========  AVERAGE =========== 
    [0.0765272,0.00148802,-0.00165705] tetas 0.00583192  0.0416507  0.00916005
    DispTr=7.49798e-10 DispMat=1.22368e-07
\end{verbatim}

Note that here we have imposed a very stric constraint on the rigidity, the final sigma being
$1e-7$ for translation and $1e-6$ for rotation. As probably this amateur camera is not completely
rigid, this explain why the residual have grown from $0.72$ tp $1.07$.


\subsubsection{Global with attachment to known value }

Sometime we want to impose that the value $r_\beta$ and $c_\beta$ stay close
to their initial values  $r^0_\beta$ and $c^0_\beta$ .
How close the camera are equals to these common value is given by the a priori
variance that we have to indicate  $\sigma^0_{Tr} $ and $\sigma^0_{\omega} $.
We the add to the bundle minimization the term :

\begin{equation}
      E^0(\sigma^0_{Tr},\sigma^0_{\omega}) =
      \sum _{\beta}
      (\frac{(r^0_\beta-r_\beta)}{\sigma^0_{\omega}}) ^2
       +(\frac{(c^0_\beta-c_\beta)}{\sigma^0 _{Tr}})^2
\end{equation}

The attachment to inial values can be specified by the {\tt OptBlocG} parameter :

\begin{itemize}
    \item  it contains to value ;
    \item  $P_1$ is $\sigma^0 _{Tr}$ and $P_2$ is $\sigma^0 _{\omega}$
    \item  if these parameter both value $-1$, then  $r_\beta$ and $c_\beta$ are
           strictly contraints to be equal to $\sigma^0_{Tr} $ and $\sigma^0_{\omega} $;
    \item  if these parameter both value $-2$,  then they are not used (may seem stupid, prepare new options).
\end{itemize}

\subsubsection{Time relative }

Some time, it may happen that the block is rigid, but with a shape that evolve smoothly accross time :
the block $k$ is close to block $k+1$ even if the final blocks can be very far from the initial one.
How close the consecutive block are is given by the a priori variance $\sigma^t_{\omega}$ and $\sigma^t_{Tr}$ 
This can be modelized mathematically by adding a term

\begin{equation}
      E^{t}(\sigma^t_{Tr},\sigma^t_{\omega}) =
      \sum _{k,\beta}
      (\frac{\Delta_{\omega}(A,\beta,k,k+1)}{\sigma^t_{\omega}}) ^2
      +(\frac{\Delta_{Tr}(A,\beta,k,k+1)}{\sigma^t_{Tr}})^2
\end{equation}

This term can be add with the {\tt BlocTimeRel} parameter, the syntax being exactly the same that 
with {\tt BlocGlob}.

\subsubsection{Combination}

When pertinent the option  {\tt BlocTimeRel} and {\tt BlocGlob} can be used 
simultaneously, for example to modelize a global evolution linked to stay close
to an initial known value.






